% Table numbers of specimens in datasets

\begin{longtable}{ccccc}

\caption{Numbers of species and specimens in our dataset.}\\ 
  
\hline
\multicolumn{5}{c}{\textbf{All specimens}}\\
  \hline
  & \multicolumn{2}{c}{\textbf{AMPHIBIANS}} & \multicolumn{2}{c}{\textbf{REPTILES}} \\
  \hline
  & \textbf{N specimens} & \textbf{\%} & \textbf{N specimens} & \textbf{\%}\\
  \hline
  Total & 2,753,451 & NA & 3,183,806 & NA\\
  Unsexed & 2,670,874 & 97 & 2,989,308 & 93.89\\
  Female & 31,905 & 1.16 & 90,296   & 2.84\\
  Male & x & 50,672 & x & 104,202\\
  \hline

\multicolumn{5}{c}{\textbf{Sexed specimens from species/genera with $\geq$ 10 specimens}}\\
  \hline
  & \multicolumn{2}{c}{\textbf{AMPHIBIANS}} & \multicolumn{2}{c}{\textbf{REPTILES}} \\
  \hline
  & \textbf{N specimens} & \textbf{\%} & \textbf{N specimens} & \textbf{\%}\\
\hline
Total & 75,233 & NA & 183,285 & NA\\
Female & 28,963 & 38.50 & 85,300 & 46.54\\
Male & 46,270 & 61.50 & 97,985 & 47.83\\
\hline

\label{table_numbers}
\end{longtable}





